\begin{titlepage}
  \begin{center}

  {\Huge AXIS\_1553\_ENCODER}

  \vspace{25mm}

  \includegraphics[width=0.90\textwidth,height=\textheight,keepaspectratio]{img/AFRL.png}

  \vspace{25mm}

  \today

  \vspace{15mm}

  {\Large Jay Convertino}

  \end{center}
\end{titlepage}

\tableofcontents

\newpage

\section{Usage}

\subsection{Introduction}

\par
AXIS 1553 Encoder is a core for taking AXIS data and encoding for output to the PMOD1553 device. The output
is a TTL differential signal. Meaning when diff[0] is 1 diff[1] is 0. This core also includes a diff enable
which allows for mux switching to the transmit (encoder) when active.

\subsection{Dependencies}

\par
The following are the dependencies of the cores.

\begin{itemize}
  \item fusesoc 2.X
  \item iverilog (simulation)
  \item cocotb (simulation)
\end{itemize}

\input{src/fusesoc/depend_fusesoc_info.tex}

\subsection{In a Project}
\par
Connect the device to your AXIS bus. TUSER is used to set various options such as command/data packet mode.

\par
TDATA input should contain the 16 bit data payload. TUSER is a 8 bit command register
that takes a discription what type of data it is (command or data) and other options.
described below.

TUSER = {TYY,NA,D,I,P} (7 downto 0)
\begin{itemize}
\item TYY = TYPE OF DATA
  \begin{itemize}
    \item 000 N/A
    \item 001 REG (NOT IMPLIMENTED)
    \item 010 DATA
    \item 100 CMD/STATUS
  \end{itemize}
  \item NA = RESERVED FOR FUTURE USE.
  \item D = DELAY ENABLED
  \begin{itemize}
    \item 1 = 4 us delay enabled.
    \item 0 = no delay between transmissions.
  \end{itemize}
  \item I = INVERT DATA
  \begin{itemize}
    \item 1 = Invert data.
    \item 0 = Normal data.
  \end{itemize}
  \item P = PARITY
  \begin{itemize}
    \item 1 = ODD
    \item 0 = EVEN
  \end{itemize}
\end{itemize}

\section{Architecture}
\par
This core is made up of a single module.
\begin{itemize}
  \item \textbf{axis\_1553\_encoder} Interface AXIS to PMOD1553 device (see core for documentation).
\end{itemize}

For register documentation please see up\_uart in \ref{Module Documentation}

\section{Building}

\par
The AXIS 1553 Encoder is written in Verilog 2001. It should synthesize in any modern FPGA software. The core comes as a fusesoc packaged core and can be
included in any other core. Be sure to make sure you have meet the dependencies listed in the previous section.

\subsection{fusesoc}
\par
Fusesoc is a system for building FPGA software without relying on the internal project management of the tool. Avoiding vendor lock in to Vivado or Quartus.
These cores, when included in a project, can be easily integrated and targets created based upon the end developer needs. The core by itself is not a part of
a system and should be integrated into a fusesoc based system. Simulations are setup to use fusesoc and are a part of its targets.

\subsection{Source Files}

\subsubsection{fusesoc\_info File List}
\begin{itemize}
\item src
	\begin{itemize}
	\item src/axis\_1553\_encoder.v
	\end{itemize}
\item tb
	\begin{itemize}
	\item {'tb/tb\_1553\_enc.v': {'file\_type': 'verilogSource'}}
	\end{itemize}
\item tb\_cocotb
	\begin{itemize}
	\item {'tb/tb\_cocotb.py': {'file\_type': 'user', 'copyto': '.'}}
	\item {'tb/tb\_cocotb.v': {'file\_type': 'verilogSource'}}
	\end{itemize}
\end{itemize}


\subsection{Targets}

\subsubsection{fusesoc\_info Targets}
\begin{itemize}
\item default
	\begin{itemize}
	\item[$\space$] Info: Default for IP intergration.
	\end{itemize}
\item sim
	\begin{itemize}
	\item[$\space$] Info: Simulation using icarus as the default.
	\end{itemize}
\end{itemize}


\subsection{Directory Guide}

\par
Below highlights important folders from the root of BUS UART.

\begin{enumerate}
  \item \textbf{docs} Contains all documentation related to this project.
    \begin{itemize}
      \item \textbf{manual} Contains user manual and github page that are generated from the latex sources.
    \end{itemize}
  \item \textbf{src} Contains source files for the core
  \item \textbf{tb} Contains test bench files for iverilog and cocotb
    \begin{itemize}
      \item \textbf{cocotb} testbench files
    \end{itemize}
\end{enumerate}

\newpage

\section{Simulation}
\par
There are a few different simulations that can be run for this core.

\subsection{iverilog}
\par
iverilog is used for simple test benches for quick verification, visually, of the core.

\subsection{cocotb}
\par
Future simulations will use cocotb. This feature is not yet implemented.

\newpage

\section{Module Documentation} \label{Module Documentation}

\begin{itemize}
\item \textbf{axis\_1553\_encoder} Interfaces AXIS to the PMOD1553.\\
\end{itemize}
The next sections document the module in great detail.

